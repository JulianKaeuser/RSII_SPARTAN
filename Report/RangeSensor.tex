 \chapter{Range Sensor}
 \label{cha:rangeSensor}

The SR02 is a distance sensor based on ultrasound, which 
allows the measurement of distances between 15 centimeters and approximately
six meters. It has an $I^{2}C$ interface, with which it can be started and read 
out. In this chapter, the communication with the sensor via this interface,
as well as the processing of the measured data, is described.

\section{Communication via $I^{2}C$}
\label{sec:sensorI2C}

here some stuff on i2c


\section{Measurement Processing}
\label{sec:filter}

Every measurement performed with sensors is subject to fluctuations, and 
may additionally oly be taken as valid if the values are within the sensors
measuring range. Hence, it is common to apply a processing procedure to 
correct erroneous values. Our filtering procedure is based on two steps, 
as presented in the following.

\subsection{Sensor Range Check}
\label{subsec:sensorRangeCheck}

The SRF02 sensor features a minimum measurable distance of 15 centimeters, 
while the maximum distance is about six meters. Consequently, any value
below the lower threshold or higher than the upper range limit is may not 
be interpreted as valid. Before any further processing, these conditions
are checked; if the currently processed value exceeds the range, it is ignored.
for further processing, the previously stored value from the last measurement 
is used.

\subsection{Median Filter}
\label{subsec:medianFilter}
Before a filter algorithm for measurements can be designed, the unfiltered
data must be examined for errors. Then, an algorithm capable of reducing 
these errors can be developed. 

In our case, we observed two noteable characteristics:

\begin{enumerate}
    \item \textbf{Rogue Results}\hfill \\
    In some cases, especially if the distance is near the lower bound of the 
    sensors range, measured values are wrong because they are much too high.
    For example, when measuring distances between 15 and 20 centimeters, some
    returned values from the sensor are in the range of 200 plus centimeters.
    These have to be filtered out only when the previously measured distances
    clearly indicate that the value around 200 cm cannot be valid.
    \item \textbf{No Distinction at High Error Rate}\hfill \\
    If many measurements are wrong (e.g. below the threshold), it is not possible 
    to determine whether the measured value is right or wrong any more. Hence, the 
    processing can never filter out all errors.
\end{enumerate}

Thus, a filter which stores a number of previous values and determines the 
validity of the current value based on these must be implemented, alothough it can 
never be perfect with this behaviour. A \emph{median filter} is suited well because 
it is capable of sorting out too high or too low, rogue measurements. The filter is 
based on an array, which stores the last $N$ unfiltered values, where $N$ is a 
parameter which must be found fo roptimal filter behaviour. 

For the median filter, three steps are necessary:
\begin{itemize}
    \item First, the oldest value in the array has to be deleted. In our 
    implementation, the oldest value is overwritten by the current measured
    value, and the buffer for the last $N$ values is a ring buffer. Consequently,
    the pointer to the oldest value is incremented, and, if it arrives at the end,
    reset to the initial value.
    \item Second, the buffer with the last $N$ values must be sorted. We chose
    the \emph{Simple Sort} algorithm, which is of $\mathcal{O}(n^{2})$ complexity,
    iterative and works in-place. SpartanMC does not allow too much recursion,
    so an iterative approach is required. The complexity is acceptable for 
    not too large buffers (see section \ref{subsubsec:filterDepth} for 
    details on the choice). As Simple Sort changes the array it is given as 
    input, we decided to copy the buffer array and sort the copy; otherwise,
    the tracking of the order of the last values would be more complex. The 
    complexity of the filter is not changed by the copying process.
    \item At last, the median value must be extracted. If the buffer length is
    odd, the value at \texttt{sorted\_array[length/2 +1]} is used; else, the
    value \texttt{sorted\_array[length/2]} is the median and thus returned.
\end{itemize}

All three steps are performed by separate functions.

\subsubsection{Choice of the Filter Size}
\label{subsubsec:filterDepth}
The number of values taken into account to determine the current filtered 
value is the filter size. Its choice affects the memory demand of the filter,
the computation time per measurement and the quality of the output. 

For our implementation of the median filter, the runtime can be neglected,
 since it will never be longer than the sampling interval of 65 ms. The 
 processor is also not concerned with other demanding tasks.

In our case, two arrays (the one with the measured values and the sorted array)
of the length of the filter have to be allocated. Therefore, the memory demand 
is linear with the filter size.

Finally, the resulting quality of the filter is the key factor for the choice
of its size. With too few previous values, the filter is not able to
neglect many erroneous measurements. A longer filter delays the output of
the corrected value too long, because for every actual larger change of
the distance, more old values have to be replaced. Thus, a larger filter is 
slower; for the median filter, the adaption to a change is even slower than
for e.g. a moving average filter.

after the consideration of these aspects and experiments with the filter size,
we chose a size of 8. In comparison with the unfiltered values, this value
produced acceptably fast changes, while effectively filtering wrong measurements.







