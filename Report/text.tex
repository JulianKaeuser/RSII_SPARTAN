\chapter{Text in Latex}
\label{cha:text}

In Latex gibt es f�r einige Symbole bestimmte Befehle, die man eingeben muss, um sie als Text richtig darstellen zu k�nnen. Im folgenden werden einige Befehle bzw. Kodierungen zur Erstellung einiger Symbole erl�utert. 


%% Umlaute Seite %%%%%%%%%%%%%%%%%%%%%%%%%%%%%%%%%%%%%%%%%%%%%%%%%%%%%%%%%%%%%%%%
\section{Umlaute und �hnliches}
\label{sec:umlaute}

Es gibt einige Methoden Umlaute in Latex zu setzten. Die einfachste erfolgt mit Hilfe eines Anf�hrungszeichens, das vor dem gew�nschten Vokal positioniert werden muss, d.h. um das Wort ``K�he'' richtig darzustellen muss man im Editor ``K''uhe'' schreiben. Genauso werden auch die Umlaute "a und "o erzeugt.\\F�r scharfes s `` "s '' braucht man auch vor einem ``s'' wieder ein Anf�hrungszeichen zu setzten.
\\
Die Codierung bzw. entsprechenden Befehle zu weiteren Sonderzeichen kann man unter \href{[http://de.wikibooks.org/wiki/LaTeX-Kompendium:_Sonderzeichen]}{www.wikibooks.org} finden.\\
\\

\section{Querverweis}
\label{sec:ref}
Innerhalb eines Textes kann man einen Querverweis einf�gen. Dies funktioniert mit dem Befehl {\textbackslash ref \{Name\}}. Der Befehl erzeugt einen Querverweis auf eine Textstelle, die zuvor durch einen {\textbackslash label}-Befehl mit dem angegebenen Namen versehen wurde. Der Querverweis gibt die Gliederungsnummer der betreffenden Textstelle an. Aus diesem Grund es ist sinnvoll Kapiteln bzw. Unterkapiteln auf die man oft verweisen m�chte mit dem Befehl {\textbackslash label\{Name\} zu vermerken. 
 
 \paragraph{Beispiel}
 
 Siehe Kapitel \ref{cha:text} um ausf�hrliche Information zum Text in Latex.
 
 In diesem Beispiel wurde auf den Kapitel Text in Latex verwiesen mit {\textbackslash ref\{text\}} da das Kapitel Text in Latex mit dem Befehl {\textbackslash label\{cha:text\}} versehen wurde. Der Befehl {\textbackslash label} muss direkt nach dem Befehl f�r den Gliederungsabschnitt bzw. �berschrift erfolgen.
 
 